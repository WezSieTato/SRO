\chapter{Testy}

\section{Plan testów}


\subsection[Test protokołu]{Test protokołu}
\begin{enumerate}
\item Sprawdzenie poprawności przesyłanej ramki danych z węzła wewnętrznego do zewnętrznego oraz z zewnętrznego do wewnętrznego.
\item Sprawdzenie poprawności przesyłanej ramki danych z węzła zewnętrznego do klienta oraz od klienta do węzła zewnętrznego.
\end{enumerate}
	
\textit{Testy obejmują sprawdzenie poprawności serializacji oraz deserializacji danych przesyłanych między elementami systemu. Sprawdzone również zostanie połączenie między tymi elementami.}

\subsubsection*[Przebieg testów]{Przebieg testów}
\begin{enumerate}
\item zbudowanie wiadomości z danymi z serwera wewnętrznego/zewnętrznego do serwera zewnętrznego/wewnętrznego RSOMessage.
\item wysłanie tablicy bajtów do serwera zewnętrznego 
\item sprawdzenie poprawności przesłanych danych
\end{enumerate}


\subsection[Test bazy danych]{Test bazy danych}
\begin{enumerate}
\item Sprawdzenie połączenia z bazą danych.
\item Sprawdzenie poprawności działania algorytmu aktualizacji danych systemu. \\
\textit{Test obejmuje sprawdzenie algorytmu wykorzystującego Token Ring do aktualizacji danych. Weryfikacji zostanie podany również konfigurowalny czas posiadania tokena przez węzeł.}
\item Sprawdzenie redundancji danych.
\end{enumerate}



\subsection[Test przetwarzania danych]{Test przetwarzania danych}
Sprawdzenie czy węzeł zewnętrzny poprawnie przetwarza dane wrażliwe na dane gotowe do wysłania klientowi. Wysyłane dane powinny zawierać numer identyfikacyjny oraz listę przedmiotów.

\subsection[Test algorytmu Heartbeatów]{Test algorytmu Heartbeatów}
Sprawdzenie czy węzły zewnętrzne poprawnie komunikują się między sobą oraz czy poprawnie interpretowane są braki komunikacji.

\subsubsection*[Przebieg testów]{Przebieg testów}
Sprawdzenie działania algorytmu dla interwału czasowego równego: 5, 30 oraz 60 sekund:
\begin{enumerate}
\item dodanie trzech węzłów zewnętrznych
\item wysłanie wiadomości Heartbeat
\item oczekiwanie całego interwału na Heartbeat zwrotny
\item (Przypadek 1) Heartbeat przyszedł, zerowanie timera
\item (Przypadek 2) Heartbeat nie przyszedł, wysłanie TestRequesta
\item (Przypadek 2.1) Heartbeat przyszedł, zerowanie timera
\item (Przypadek 2.2) Heartbeat nie przyszedł, uznanie węzła za nieaktywny
\end{enumerate}

\subsection[Testy połączeniowe węzłów]{Testy połączeniowe węzłów}
\begin{enumerate}
\item Test podłączenia oraz odłączenia węzła wewnętrznego
\begin{enumerate}
\item Podłączenie nowego węzła przy braku aktywnych innych węzłów wewnętrznych
	\begin{enumerate}
	\item włączenie nowego węzła
	\item sprawdzenie czy węzły zewnętrzne komunikują się z nowym węzłem
	\end{enumerate}
\item Podłączenie nowego węzła gdy przynajmniej jeden węzeł wewnętrzny jest aktywny 
	\begin{enumerate}
	\item włączenie nowego węzła
	\item sprawdzenie czy działa przekazanie tokena w warstwie wewnętrznej
	\end{enumerate}
\end{enumerate}

\item Test podłączenia oraz odłączenia węzła zewnętrznego
\begin{enumerate}
\item Podłączenie nowego węzła przy braku aktywnych innych węzłów zewnętrznych
\item Podłączenie nowego węzła gdy przynajmniej jeden jest węzeł zewnętrzny aktywny
\item Podłączenie z węzłem wewnętrznym

\end{enumerate}
\end{enumerate}

Scenariusze testowe mają za zadanie sprawdzenie zachowania systemu podczas próby podłączenia nowego węzła w zależności od ilości elementów w danej warstwie oraz zachowanie systemu w zależności od ilości jego elementów.
