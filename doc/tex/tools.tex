\chapter{Narzędzia i zewnętrzne biblioteki}

\section{Docker}

\subsection[Czym jest docker?]{Czym jest docker?}
\par{Docker jest narzędziem przeznaczonym do tworzenie przenośnych, wirtualnych kontenerów pozwalających na prostą i szybką replikację środowiska.}

\par{Oprogramowanie Docker wprowadza standaryzację w środowisko uruchomieniowe. Generowane kontenery są spójne i takie same w różnych środowiskach. W chwili obecnej Docker wymaga jądra Linux do uruchomienia, jednak działa nie tylko w wersji natywnej, ale również poprzez wirtualizację specjalnych minimalistycznych dystrybucji Linux na systemach Mac OS X oraz Windows (na przykład w darmowym narzędziu do wirtualizacji – VirtualBox).}

\subsection[Spójność]{Spójność}

\par{Docker wprowadzenie spójność w środowisko deweloperskie. Często, kiedy nad projektem pracuje więcej niż jeden programista (co w dzisiejszych czasach jest normą) pojawia się problem z różnymi wersjami użytego oprogramowania. Zdarza się, że jedna wersja biblioteki zachowuje się inaczej od drugiej (na przykład z powodu błędu). Kontener z założenia posiada jedną, konkretną wersję każdej biblioteki koniecznej do uruchomienie aplikacji. Zwiększa to przewidywalność oprogramowania. Oczekujemy bowiem, że w tym samym środowisku, aplikacja będzie zachowywać się tak samo. }

\subsection[Cykl Dockera]{Cykl Dockera}


\begin{enumerate}
\item Stworzenie kontenera wraz z wszelkimi narzędziami i bibliotekami koniecznymi do uruchomienia aplikacji. 
\item Rozprowadzenie kontenera, wraz ze wszystkimi narzędziami i bibliotekami. 
\item Uruchomienie identycznego kontenera na dowolnej liczbie węzłów. 
\end{enumerate}

\par{Prowadzi to do znacznych ułatwień w tworzeniu oprogramowania. Ten sam kontener może zostać rozprowadzony pomiędzy deweloperami, testerami, serwerami ciągłej integracji i w końcu środowiskiem produkcyjnym. }

\subsection[Centralne repozytorium]{Centralne repozytorium}

\par{Każdy zarejestrowany użytkownik ma możliwość wgrywania własnych kontenerów do centralnego, ogólnie dostępnego repozytorium. Można tam znaleźć wiele różnych gotowych obrazów do pobrania. Są one przygotowane zarówno przez społeczność jak i przez twórców Dockera. }

\par{Centralne repozytorium pozwala użytkownikom na pobranie interesujących obrazów, szczególnie warte uwagi są repozytoria przygotowane pod konkretne rozwiązania (na przykład specjalnie pod serwer baz danych MySQL lub mongoDB). }

\par{Docker umożliwia również stworzenie prywatnych repozytoriów. Dzięki temu nie musimy upubliczniać prywatnych obrazów, ale jednocześnie możemy je rozprowadzać. Inną dostępną metodą jest zapis obrazu do pliku i przekazanie go w tradycyjny sposób. }


\section{Biblioteka ZeroMQ}

\section{Biblioteka protobuf}