
%\documentclass[a4paper,polish,onecolumn,oneside,floatssmall,11pt,titleauthor,wide,openright]{mwrep}
%\usepackage[scale={0.7,0.8},paper=a4paper,twoside]{geometry}
\documentclass[a4paper,onecolumn,oneside,11pt,wide,floatssmall]{mwrep}
\usepackage[utf8]{inputenc}
\usepackage[T1]{fontenc}
\usepackage[MeX]{polski}
\usepackage{amsmath}
\usepackage{amsfonts}
\usepackage{amssymb}
\usepackage{amsthm}
\usepackage{tgpagella}
\usepackage{geometry}
%\usepackage[pdftex, bookmarks=true, bookmarksnumbered=true, colorlinks, linkcolor=black]{hyperref}
\usepackage{color}
\usepackage{listings}
\usepackage[final]{graphicx}
\graphicspath{ {../img/tytulowa/}  {../img/diagramy/} {../img/architektura/}}
\def\url#1{{ \tt #1}}


% Mojae include !!!!!!!!!!!!!!!!
\usepackage{dirtree}
\usepackage{tabularx}
\usepackage{hyperref}
\usepackage{pdflscape}

% glebokie listy
\usepackage{enumitem}
\newlist{longenum}{enumerate}{6}
\setlist[longenum,1]{label=\roman*)}
\setlist[longenum,2]{label=\alph*)}
\setlist[longenum,3]{label=\arabic*)}
\setlist[longenum,4]{label=(\roman*)}
\setlist[longenum,5]{label=(\alph*)}

% glebokiue spisy
%\linespread{1.3}
\setcounter{secnumdepth}{3}
\setcounter{tocdepth}{3}

%\usepackage[backend=bibtex]{biblatex}

% marginesy
\textwidth\paperwidth
\advance\textwidth -55mm
\oddsidemargin-0.9in
\advance\oddsidemargin 33mm
\evensidemargin-0.9in
\advance\evensidemargin 33mm
\topmargin -1in
\advance\topmargin 25mm
\setlength\textheight{48\baselineskip}
\addtolength\textheight{\topskip}
\marginparwidth15mm

\clubpenalty=10000 % to kara za sierotki
\widowpenalty=10000 % nie pozostawia wdów
\brokenpenalty=10000 % nie dzieli wyrazów pomiędzy stronami
\sloppy

\tolerance4500
\pretolerance250
\hfuzz=1.5pt
\hbadness1450

% ŻYWA PAGINA
\renewcommand{\chaptermark}[1]{\markboth{\scshape\small\bfseries \
#1}{\small\bfseries \ #1}}
\renewcommand{\sectionmark}[1]{\markboth{\scshape\small\bfseries\thesection.\
#1}{\small\bfseries\thesection.\ #1}}
\newcommand{\headrulewidth}{0.5pt}
\newcommand{\footrulewidth}{0.pt}
\pagestyle{uheadings}

%Symbole używane w listach
\renewcommand{\labelitemi}{$\bullet$}
\renewcommand{\labelitemii}{--}
\renewcommand{\labelitemiii}{$\circ$}




% \textheight232mm
% \setlength{\textwidth}{\textwidth}
% \setlength{\oddsidemargin}{\evensidemargin}
% \setlength{\evensidemargin}{0.3cm}
\usepackage[sort, compress]{cite}

%\usepackage{multibib}
%\newcites{bk,st,doc,web}{Książki i~artykuły,Standardy i~zalecenia,Dokumentacja produktów,Publikacje i~serwisy internetowe}

\theoremstyle{definition}
\newtheorem{defn}{Definicja}[section]
\newtheorem{conj}{Teza}[section]
\newtheorem{conjmain}{Teza}
\newtheorem{exmp}{Przykład}[section]

\theoremstyle{plain}% default
\newtheorem{thm}{Twierdzenie}[section]
\newtheorem{lem}[thm]{Lemat}
\newtheorem{prop}[thm]{Hipoteza}
\newtheorem*{cor}{Wniosek}

\theoremstyle{remark}
\newtheorem*{rem}{Uwaga}
\newtheorem*{note}{Uwaga}
\newtheorem{case}{Przypadek}

\definecolor{ListingBackground}{rgb}{0.95,0.95,0.95}

\begin{document}

% kody źródłowe wplatane w tekst
\lstdefinestyle{incode}
{
basicstyle={\footnotesize},
keywordstyle={\bf\footnotesize\color{blue}},
commentstyle={\em\footnotesize\color{magenta}},
numbers=left,
stepnumber=5,
firstnumber=1,
numberfirstline=true,
numberblanklines=true,
numberstyle={\sf\tiny},
numbersep=10pt,
tabsize=2,
xleftmargin=17pt,
framexleftmargin=3pt,
framexbottommargin=2pt,
framextopmargin=2pt,
framexrightmargin=0pt,
showstringspaces=true,
backgroundcolor={\color{ListingBackground}},
extendedchars=true,
% title=\lstname,
captionpos=b,
% abovecaptionskip=1pt,
% belowcaptionskip=1pt,
frame=tb,
framerule=0pt,
}

% kody źródłowe z podpisem
\lstdefinestyle{outcode}
{
basicstyle={\footnotesize},
keywordstyle={\bf\footnotesize\color{blue}},
commentstyle={\em\footnotesize\color{magenta}},
numbers=left,
stepnumber=5,
firstnumber=1,
numberfirstline=true,
numberblanklines=true,
numberstyle={\sf\tiny},
numbersep=10pt,
tabsize=2,
xleftmargin=17pt,
framexleftmargin=3pt,
framexbottommargin=2pt,
framextopmargin=2pt,
framexrightmargin=0pt,
showstringspaces=true,
backgroundcolor={\color{ListingBackground}},
extendedchars=true,
% title=\lstname,
captionpos=b,
% abovecaptionskip=1pt,
% belowcaptionskip=1pt,
frame=tb,
framerule=0.1pt,
}

\renewcommand*\lstlistingname{Wydruk}
\renewcommand*\lstlistlistingname{Spis wydruków}

\pagenumbering{roman}
\renewcommand{\baselinestretch}{1.0}
\raggedbottom

\begin{titlepage}
    % Strona tytułowa
    \vbox to\textheight{\hyphenpenalty=10000
    \begin{center}
	\begin{tabular}{p{107mm} p{9cm}}
	    \begin{minipage}{9cm}
	      \begin{center}
	      Politechnika Warszawska \\
	      Wydział Elektroniki i~Technik Informacyjnych \\
	     
	      \end{center}
	    \end{minipage}
	    &
	    \begin{minipage}{8cm}
	    \begin{flushleft}
	     \footnotesize
	      Rok akademicki 2014/2015
	    \vspace*{2.75\baselineskip}
	    \end{flushleft}
	    \end{minipage} \\
	\end{tabular}
	\vspace*{3.75\baselineskip}

		
	\par\vspace{\smallskipamount}
	\vspace*{2\baselineskip}{\LARGE Dokumentacja projektu SRO\par}
	\vspace{3\baselineskip}{\strut Domagała Bartosz, Kornata Jarosław, Modzelewski Jędrzej, Marcin Stepnowski\par}
	\vspace*{2\baselineskip}{\huge\bfseries Usługa bezpiecznej niezawodnej dystrybucji przetworzonej chronionej informacji\par}

	\vspace*{7\baselineskip}
	\hfill\mbox{}\par\vspace*{\baselineskip}\noindent
	\begin{tabular}[b]{@{}p{3cm}@{\ }l@{}}
	    {\large\hfill } & {\large }
	\end{tabular}
	\hfill
	\begin{tabular}[b]{@{}l@{}}
	Prowadzący projekt: \\[\smallskipamount]
	{\large dr inż. Tomasz Jordan Kruk}
	\end{tabular}\par
	\vspace*{4\baselineskip}
    \begin{tabular}{p{\textwidth}}
    \begin{flushleft}
	\begin{minipage}{7cm}
	%\centerline{\footnotesize Podpis Przewodniczącego} \par
	%\centerline{\footnotesize Komisji Egzaminu Dyplomowego}\par
	\end{minipage}
    \end{flushleft}
    \end{tabular}
    \end{center}}

\end{titlepage}

% ex: set tabstop=4 shiftwidth=4 softtabstop=4 noexpandtab fileformat=unix filetype=tex spelllang=pl,en spell:


\tableofcontents
% \addcontentsline{toc}{chapter}{{Przedmowa1}{vii}}{vii}

% \chapter*{Spis tablic, rysunków i~wydruków}
% \listoftables
% \listoffigures
% \lstlistoflistings

%\setlength{\baselineskip}{7mm}
\newpage
\pagenumbering{arabic}
\setcounter{page}{1}

\chapter{Wstęp}
\chapter{Organizacja projektu}
\section{Podział zadań}

\par{\textbf{Kierownik projektu} - Domagała Bartosz}
\par{\textit{Odpowiedzialny za}}

\par{\textbf{Architekt} - Stepnowski Marcin}
\par{\textit{Odpowiedzialny za}}

\par{\textbf{Osoba odpowiedzialna za repozytorium} - Stepnowski Marcin}
\par{\textit{Odpowiedzialny za}}

\par{\textbf{Dokumentalista} - Domagała Bartosz}
\par{\textit{Odpowiedzialny za}}

\par{\textbf{Tester} - Jędrzej Modzelewski}
\par{\textit{Odpowiedzialny za}}

\par{\textbf{Handlowiec} - Kornata Jarosław}
\par{\textit{Odpowiedzialny za}}

\par{\textbf{Specjalista ds. Docker'a} - Kornata Jarosław}
\par{\textit{Odpowiedzialny za}}

\par{\textbf{Programiści} - Kornata Jarosław, Modzelewski Jędrzej, Stepnowski Marcin}
\par{\textit{Odpowiedzialny za}}

\section{Planowany harmonogram pracy}

\par{}

\section{Sprawozdania ze spotkań}

\par{}

\subsection*{Spotkanie I - 31 marca 2015}

\par{}

\subsubsection*{Streszczenie spotkania}

\par{}

\subsubsection*{Szczegółowy opis wykonanej pracy}

\par{}

\subsection*{Spotkanie II - 8 kwietnia 2015}

\par{}

\subsubsection*{Streszczenie spotkania}

\par{}

\subsubsection*{Szczegółowy opis wykonanej pracy}

\par{}

\appendix

% tutaj załączniki

%\chapter*{Bibliografia}
\nocite{*}


\end{document}

% ex: set tabstop=4 shiftwidth=4 softtabstop=4 noexpandtab fileformat=unix filetype=tex spelllang=pl,en spell:

