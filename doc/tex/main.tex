\pdfoutput=1
\pdfcompresslevel=9
%\documentclass[a4paper,polish,onecolumn,oneside,floatssmall,11pt,titleauthor,wide,openright]{mwrep}
%\usepackage[scale={0.7,0.8},paper=a4paper,twoside]{geometry}
\documentclass[a4paper,onecolumn,oneside,11pt,wide,floatssmall]{mwrep}
% \usepackage{polish}
\usepackage{amsmath}
\usepackage{amsfonts}
\usepackage{amssymb}
\usepackage{amsthm}
\usepackage{bookman}
\usepackage{listings}
\usepackage{array}
\usepackage{listings}
\usepackage{tabularx}
\usepackage{longtable}

\lstset{
	language=Java,
    basicstyle=\scriptsize,
    aboveskip={1.0\baselineskip},
    columns=fixed,
    showstringspaces=false,
    extendedchars=true,
    breaklines=true,
    tabsize=4,
    prebreak = \raisebox{0ex}[0ex][0ex]{\ensuremath{\hookleftarrow}},
    frame=single,
    showtabs=false,
    showspaces=false,
    showstringspaces=false,
    identifierstyle=\ttfamily,
    keywordstyle=\color[rgb]{0,0,1},
    commentstyle=\color[rgb]{0.133,0.545,0.133},
    stringstyle=\color[rgb]{0.627,0.126,0.941},
    numbers=left,
    numberstyle=\tiny,
    stepnumber=1,
    numbersep=5pt,
    captionpos=b,
    escapeinside={\%*}{*)}
}

\usepackage{geometry}
\usepackage[utf8x]{inputenc}
\usepackage[T1]{fontenc}
% \usepackage{t1enc}
% \usepackage[pdftex, bookmarks]{hyperref}
\usepackage[pdftex, bookmarks=false]{hyperref}
\def\url#1{{ \tt #1}}

\usepackage[parfill]{parskip}
\usepackage{listings}

% marginesy
\parskip = .4\baselineskip
\textwidth\paperwidth
\advance\textwidth -55mm
\oddsidemargin-0.9in
\advance\oddsidemargin 33mm
\evensidemargin-0.9in
\advance\evensidemargin 33mm
\topmargin -1in
\advance\topmargin 25mm
\setlength\textheight{48\baselineskip}
\addtolength\textheight{\topskip}
\marginparwidth15mm

\clubpenalty=10000 % to kara za sierotki
\widowpenalty=10000 % nie pozostawia wdów
\brokenpenalty=10000 % nie dzieli wyrazów pomiędzy stronami
\sloppy

\tolerance4500
\pretolerance250
\hfuzz=1.5pt
\hbadness1450

% ŻYWA PAGINA
\renewcommand{\chaptermark}[1]{\markboth{\scshape\small\bfseries \
#1}{\small\bfseries \ #1}}
\renewcommand{\sectionmark}[1]{\markboth{\scshape\small\bfseries\thesection.\
#1}{\small\bfseries\thesection.\ #1}}
\newcommand{\headrulewidth}{0.5pt}
\newcommand{\footrulewidth}{0.pt}
\pagestyle{uheadings}

\usepackage[pdftex]{color,graphicx}
\usepackage[polish]{babel}

% \textheight232mm
% \setlength{\textwidth}{\textwidth}
% \setlength{\oddsidemargin}{\evensidemargin}
% \setlength{\evensidemargin}{0.3cm}
\usepackage[sort, compress]{cite}

%\usepackage{multibib}
%\newcites{bk,st,doc,web}{Książki i~artykuły,Standardy i~zalecenia,Dokumentacja produktów,Publikacje i~serwisy internetowe}

\theoremstyle{definition}
\newtheorem{defn}{Definicja}[section]
\newtheorem{conj}{Teza}[section]
\newtheorem{conjmain}{Teza}
\newtheorem{exmp}{Przykład}[section]

\theoremstyle{plain}% default
\newtheorem{thm}{Twierdzenie}[section]
\newtheorem{lem}[thm]{Lemat}
\newtheorem{prop}[thm]{Hipoteza}
\newtheorem*{cor}{Wniosek}

\theoremstyle{remark}
\newtheorem*{rem}{Uwaga}
\newtheorem*{note}{Uwaga}
\newtheorem{case}{Przypadek}

\definecolor{ListingBackground}{rgb}{0.95,0.95,0.95}

\begin{document}

% kody źródłowe wplatane w tekst
\lstdefinestyle{incode}
{
basicstyle={\footnotesize},
keywordstyle={\bf\footnotesize\color{blue}},
commentstyle={\em\footnotesize\color{magenta}},
numbers=left,
stepnumber=5,
firstnumber=1,
numberfirstline=true,
numberblanklines=true,
numberstyle={\sf\tiny},
numbersep=10pt,
tabsize=2,
xleftmargin=17pt,
framexleftmargin=3pt,
framexbottommargin=2pt,
framextopmargin=2pt,
framexrightmargin=0pt,
showstringspaces=true,
backgroundcolor={\color{ListingBackground}},
extendedchars=true,
% title=\lstname,
captionpos=b,
% abovecaptionskip=1pt,
% belowcaptionskip=1pt,
frame=tb,
framerule=0pt,
}



% kody źródłowe z podpisem
\lstdefinestyle{outcode}
{
basicstyle={\footnotesize},
keywordstyle={\bf\footnotesize\color{blue}},
commentstyle={\em\footnotesize\color{magenta}},
numbers=left,
stepnumber=5,
firstnumber=1,
numberfirstline=true,
numberblanklines=true,
numberstyle={\sf\tiny},
numbersep=10pt,
tabsize=2,
xleftmargin=17pt,
framexleftmargin=3pt,
framexbottommargin=2pt,
framextopmargin=2pt,
framexrightmargin=0pt,
showstringspaces=true,
backgroundcolor={\color{ListingBackground}},
extendedchars=true,
% title=\lstname,
captionpos=b,
% abovecaptionskip=1pt,
% belowcaptionskip=1pt,
frame=tb,
framerule=0.1pt,
}

\renewcommand*\lstlistingname{Wydruk}
\renewcommand*\lstlistlistingname{Spis wydruków}

\pagenumbering{roman}
\renewcommand{\baselinestretch}{1.0}
\raggedbottom

\begin{titlepage}
    % Strona tytułowa
    \vbox to\textheight{\hyphenpenalty=10000
    \begin{center}
	\begin{tabular}{p{107mm} p{9cm}}
	    \begin{minipage}{9cm}
	      \begin{center}
	      Politechnika Warszawska \\
	      Wydział Elektroniki i~Technik Informacyjnych \\
	      Instytut Informatyki
	      \end{center}
	    \end{minipage}
	    &
	    \begin{minipage}{8cm}
	    \begin{flushleft}
	     \footnotesize
	      Rok akademicki 2013/2014
	    \vspace*{2.75\baselineskip}
	    \end{flushleft}
	    \end{minipage} \\
	\end{tabular}
	\vspace*{3.75\baselineskip}

		
	\par\vspace{\smallskipamount}
	\vspace*{2\baselineskip}{\LARGE PRACA DYPLOMOWA INŻYNIERSKA\par}
	\vspace{3\baselineskip}{\LARGE\strut Marcin Stepnowski\par}
	\vspace*{2\baselineskip}{\huge\bfseries Rozbudowa aplikacji SpriteBuilder o wsparcie dla silnika Cocos2d-x\par}

	\vspace*{7\baselineskip}
	\hfill\mbox{}\par\vspace*{\baselineskip}\noindent
	\begin{tabular}[b]{@{}p{3cm}@{\ }l@{}}
	    {\large\hfill } & {\large }
	\end{tabular}
	\hfill
	\begin{tabular}[b]{@{}l@{}}
	Opiekun pracy: \\[\smallskipamount]
	{\large mgr inż. Waldemar Grabski}
	\end{tabular}\par
	\vspace*{4\baselineskip}
    \begin{tabular}{p{\textwidth}}
    \begin{flushleft}
	\begin{minipage}{7cm}
	Ocena \dotfill
	\par\vspace{1.6\baselineskip}
	\dotfill
	\par\noindent
	\centerline{\footnotesize Podpis Przewodniczącego} \par
	\centerline{\footnotesize Komisji Egzaminu Dyplomowego}\par
	\end{minipage}
    \end{flushleft}
    \end{tabular}
    \end{center}}

    % Życiorys
    \newpage\thispagestyle{empty}
    \begin{tabular}{p{5cm} p{12cm}}
    \begin{minipage}{5cm}
    \center
    \end{minipage}
    &
    \begin{minipage}{12cm}
    \begin{flushleft}
    \par\noindent\vspace{1\baselineskip}
    \renewcommand{\arraystretch}{1.5}\begin{tabular}[h]{l l}
    {\normalsize\it Kierunek:} & Informatyka \\
    {\normalsize\it Specjalność:} & Inżynieria Systemów Informatycznych \\
    \end{tabular}
    \par\noindent\vspace{1\baselineskip}
    \renewcommand{\arraystretch}{1.5}\begin{tabular}[h]{l l}
    {\normalsize\it Data urodzenia:} & {\normalsize 31 stycznia 1991~r.} \\
    {\normalsize\it Data rozpoczęcia studiów:} & {\normalsize 21 lutego 2011 r.}
    \end{tabular}
    \par\noindent\vspace{1\baselineskip}
    \end{flushleft}
    \end{minipage}
    \end{tabular}
    \vspace*{1\baselineskip}
    \begin{center}
	{\large\bfseries Życiorys}\par\bigskip
    \end{center}

    \indent
    
\begin{small}
    Urodziłem się 31 stycznia 1991 r. w Ostrołęce. W latach 2008-2010 uczęszczałem do II LO im. Kamila Cypriana Norwida w Ostrołęce, do klasy o profilu matematyczno-informatyczno-językowym. W lutym 2011 r. rozpoczączełem studia dzienne pierwszego stopnia na kierunku Informatyka na Wydziale Elektroniki i Technik Informacyjnych Politechniki Warszawskiej. 
    
    W lipcu 2012 podjąłem pracę jako programista aplikacji mobilnych w Fundacji Festina Lente. Współpracowałem także z Instytutem Badań Stosowanych Politechniki Warszawskiej tworząc aplikację mobilną na systemy iOS. Dziedzinami wiedzy, które najbardziej mnie pasjonują na moim kierunku są gry komputerowe, aplikacje mobilne, wzorce projektowe oraz stosowanie przejrzystego stylu implementacji.
    Poza studiami i pracą interesuję się muzyką rockową z lat 70 oraz prowadzę małą pasiekę pszczelarską umiejscowioną w Aleksandrowie, niedaleko Ostrołęki.
\end{small}
    
    \par
    \vspace{2\baselineskip}
    \hfill\parbox{15em}{{\small\dotfill}\\[-.3ex]
    \centerline{\footnotesize podpis studenta}}\par
    \vspace{3\baselineskip}
    \begin{center}
 	{\large\bfseries Egzamin dyplomowy} \par\bigskip\bigskip
    \end{center}
    \par\noindent\vspace{1.5\baselineskip}
    Złożył egzamin dyplomowy w dn. \dotfill
    \par\noindent\vspace{1.5\baselineskip}
    Z wynikiem \dotfill
    \par\noindent\vspace{1.5\baselineskip}
    Ogólny wynik studiów \dotfill
    \par\noindent\vspace{1.5\baselineskip}
    Dodatkowe wnioski i uwagi Komisji \dotfill
    \par\noindent\vspace{1.5\baselineskip}
    \dotfill

    % Streszczenie
    \newpage\thispagestyle{empty}
    \vspace*{2\baselineskip}
    \begin{center}
	{\large\bfseries Streszczenie}\par\bigskip
    \end{center}

    {\itshape
    Niniejsza praca poświęcona jest opracowaniu rozszerzenia dla programu SpriteBuilder, które pozwoliłoby na importowanie stworzonego graficznego interfejsu użytkownika do kodu źródłowego w języku C++ z wykorzystaniem silnika do tworzenia gier Cocos2d-x. Program ten jest dostępny na platformę Mac OS X, jednak generowany możemy skompilować i uruchomić na większości obecnie dostępnych systemach operacyjnych (iOS, Android, Windows, Windows Phone, Linux, Mac OS X, a także inne). Praca została zaprojektowana z myślą o łatwym i prostym wprowadzaniu zmian do projektu w przyszłości. }
    \vspace*{1\baselineskip}

    \noindent{\bf Słowa kluczowe}: {\itshape graficzny interfejs użytkownika, silnik gry, edytor}
    \par
    \vspace{4\baselineskip}
    \begin{center}
	{\large\bfseries Abstract}\par\bigskip
    \end{center}
    \noindent{\bf Title}: {\itshape Adding support for cocos2d-x game engine to SpriteBuilder app}\par
    \vspace*{1\baselineskip}
    {\itshape
    The aim of this thesis is elaborate the extension of SpireBuilder aplication, which allow you to import just created graphical user interface to the source code in C++ with using game engine Cocos2d-x. This program is accessible on Mac OS X platform, however generated code we can compile and run on currently available operation systems (iOS, Android, Windows, Windows Phone, Linux, Mac OS C and more).
    }
    \vspace*{1\baselineskip}

    \noindent{\bf Key words}: {\itshape GUI, game engine, editor}

\end{titlepage}

% ex: set tabstop=4 shiftwidth=4 softtabstop=4 noexpandtab fileformat=unix filetype=tex spelllang=pl,en spell:


\tableofcontents
% \addcontentsline{toc}{chapter}{{Przedmowa1}{vii}}{vii}

% \chapter*{Spis tablic, rysunków i~wydruków}
% \listoftables
% \listoffigures
% \lstlistoflistings

%\setlength{\baselineskip}{7mm}
\newpage
\pagenumbering{arabic}
\setcounter{page}{1}

\chapter{Opis projektu}

\section{Temat projektu}

\section{Pomysł rozwiązania}

\section{Wymagania}

\subsection{Wymagania funkcjonalne}

\subsection{Wymagania niefunkcjonalne}

\section{Wstępny projekt architektury}

\section{Potencjalne problemy}

\section{Organizacja środowiska programistycznego}
\chapter{Testy}

\section{Plan testów}


\subsection[Test protokołu]{Test protokołu}
\begin{enumerate}
\item Sprawdzenie poprawności przesyłanej ramki danych z węzła wewnętrznego do zewnętrznego oraz z zewnętrznego do wewnętrznego.
\item Sprawdzenie poprawności przesyłanej ramki danych z węzła zewnętrznego do klienta oraz od klienta do węzła zewnętrznego.
\end{enumerate}
	
\textit{Testy obejmują sprawdzenie poprawności serializacji oraz deserializacji danych przesyłanych między elementami systemu. Sprawdzone również zostanie połączenie między tymi elementami.}

\subsection[Test bazy danych]{Test bazy danych}
\begin{enumerate}
\item Sprawdzenie połączenia z bazą danych.
\item Sprawdzenie poprawności działania algorytmu aktualizacji danych systemu. \\
\textit{Test obejmuje sprawdzenie algorytmu wykorzystującego Token Ring do aktualizacji danych. Weryfikacji zostanie podany również konfigurowalny czas posiadania tokena przez węzeł.}
\item Sprawdzenie redundancji danych.

\end{enumerate}

\subsection[Test przetwarzania danych]{Test przetwarzania danych}
Sprawdzenie czy węzeł zewnętrzny poprawnie przetwarza dane wrażliwe na dane gotowe do wysłania klientowi. Wysyłane dane powinny zawierać numer identyfikacyjny oraz listę przedmiotów.

\subsection[Test algorytmu Heartbeatów]{Test algorytmu Heartbeatów}
Sprawdzenie czy węzły zewnętrzne poprawnie komunikują się między sobą oraz czy poprawnie interpretowane są braki komunikacji.

\subsection[Testy połączeniowe węzłów]{Testy połączeniowe węzłów}
\begin{enumerate}
\item Test podłączenia oraz odłączenia węzła wewnętrznego
\begin{enumerate}
\item Podłączenie nowego węzła przy braku aktywnych innych węzłów wewnętrznych
\item Podłączenie nowego węzła gdy przynajmniej jeden węzeł wewnętrzny jest aktywny 
\end{enumerate}

\item Test podłączenia oraz odłączenia węzła zewnętrznego
\begin{enumerate}
\item Podłączenie nowego węzła przy braku aktywnych innych węzłów zewnętrznych
\item Podłączenie nowego węzła gdy przynajmniej jeden jest węzeł zewnętrzny aktywny
\item Podłączenie z węzłem wewnętrznym

\end{enumerate}
\end{enumerate}

Scenariusze testowe mają za zadanie sprawdzenie zachowania systemu podczas próby podłączenia nowego węzła w zależności od ilości elementów w danej warstwie oraz zachowanie systemu w zależności od ilości jego elementów.

\chapter[Organizacja projektu]{Organizacja projektu}
\section[Podział zadań]{Podział zadań}

\par{\textbf{Kierownik projektu, Dokumentalista} - Domagała Bartosz}

\par{\textit{Odpowiedzialny za planowanie, realizację oraz zamykanie projektu. Ma zapewnić osiągnięcie założonych celów projektu i wytworzenie oprogramowania spełniającego określone wymagania jakościowe. Ponadto odpowiedzialny za całą dokumentację projektu, nadzorujący jej tworzenie i ostateczną formę.}}



\par{\textbf{Handlowiec, Specjalista ds. Docker'a, Programista} - Kornata Jarosław}
\par{\textit{Odpowiedzialny za prezentację projektu przed prowadzącym. Ponadto główny specjalista rozwiązania Docker, posiadający na ten temat największą wiedzę, przekazywaną w trakcie projektu pozostałym osobom. Bierze czynny udział w tworzeniu kodu źródłowego programu oraz sumiennie dokumentuje swoją pracę.}}


\par{\textbf{Tester, Programista} - Jędrzej Modzelewski}
\par{\textit{Odpowiedzialny za wszelkie testy związane z oprogramowaniem - projektujący je, programujący oraz egzekwujący. Bierze czynny udział w tworzeniu kodu źródłowego programu oraz sumiennie dokumentuje swoją pracę.}}

\par{\textbf{Architekt, Osoba odp. za repozytorium, Programista} - Stepnowski Marcin}
\par{\textit{Odpowiedzialny za stworzenie i dbanie o odpowiednie wdrożenie architektury projektu oraz dba o zgodność tworzonych rozwiązań informatycznych z obowiązującymi standardami, wzorcami i strategią. Ponadto odpowiedzialny za założenie, zarządzanie oraz tworzenie kopii zapasowych repozytorium projektu. Bierze czynny udział w tworzeniu kodu źródłowego programu oraz sumiennie dokumentuje swoją pracę.}}


\section[Planowany harmonogram pracy]{Planowany harmonogram pracy}

\par{}

\section[Sprawozdania ze spotkań]{Sprawozdania ze spotkań}

\par{Sprawozdania z konkretnych spotkań zawierają listę ustaleń oraz wykonanych czynności w związku z projektem. Poza spotkaniami, każdy z uczestników projektu pracował nad przydzielonymi zadaniami indywidualnie, we własnym zakresie, trzymając się konkretnych terminów.}

\subsection[Spotkanie I - 31 marca 2015]{Spotkanie I - 31 marca 2015}

\begin{itemize}
\item Domagała Bartosz - \textit{obecny}
\item Kornata Jarosław - \textit{obecny}
\item Modzelewski Jędrzej - \textit{obecny}
\item Stepnowski Marcin - \textit{obecny}
\end{itemize}

\par{Spotkanie organizacyjne, którego celem był podział ról w projekcie oraz przydzielenie zadań, a także wstępne omówienie samej wizji projektu.}


\subsubsection*[Szczegółowy opis wykonanej pracy]{Szczegółowy opis wykonanej pracy}

\begin{itemize}
\item Kierownik projektu ustalił podział ról zgodnie z umiejętnościami i preferencjami osób z grupy projektowej:

\begin{itemize}
\item Domagała Bartosz - Kierownik, Dokumentalista
\item Kornata Jarosław - Handlowiec, Specjalista ds. Docker'a, Programista
\item Modzelewski Jędrzej - Tester, Programista
\item Stepnowski Marcin - Architekt, Osoba odpowiedzialna za repozytorium, Programista
\end{itemize}

\item Ustalony został termin cotygodniowych spotkań projektowych
\item Omówiona została treść zadania, upewniono się, że każdy z uczestników projektu je rozumie
\item Prowadzone były rozmowy na temat wizji projektu każdego z uczestników, dobre pomysły spisywane i komentowane
\item Zastanowiono się nad sprecyzowaniem treści zadania
\item Każdy z uczestników projektu otrzymał zadania do wykonania:
\begin{itemize}
\item Domagała Bartosz - stworzenie szkieletu dokumentacji za pomocą LateXa, zapisanie sprawozdania z odbytego spotkania
\item Kornata Jarosław - dowiedzenie się jak najwięcej o rozwiązaniu Docker
\item Modzelewski Jędrzej - pomyślenie o narzędziach, bibliotekach i językach programowania jakich można użyć w projekcie
\item Stepnowski Marcin - stworzenie repozytorium Git oraz zastanowienie się nad architekturą projektu
\end{itemize}
\end{itemize}

\subsection[Spotkanie II - 8 kwietnia 2015]{Spotkanie II - 8 kwietnia 2015}

\begin{itemize}
\item Domagała Bartosz - \textit{obecny}
\item Kornata Jarosław - \textit{obecny}
\item Modzelewski Jędrzej - \textit{obecny}
\item Stepnowski Marcin - \textit{obecny}
\end{itemize}

\par{Spotkanie podsumowujące ostatnio wykonane zadania oraz występujące problemy. Ponadto podczas niego ustalono wstępną treść zadania i omówiono, a także przetestowano rozwiązanie Docker.}


\subsubsection*[Szczegółowy opis wykonanej pracy]{Szczegółowy opis wykonanej pracy}

\begin{itemize}
\item Przekazanie przez specjalistę ds. Dockera informacji o tym rozwiązaniu pozostałym uczestnikom
\item Wspólne sprawdzenie testowych konfiguracji Dockera
\item Wstępny projekt architektury projektu zaproponowany przez Architekta
\item Zastanowienie się nad wymaganiami niefunkcjonalnymi projektu
\item Ustalenie języka w jakim stworzony zostanie projekt - Java
\item Rozmowa na temat przydatnych i potrzebnych bibliotek w projekcie

\item Każdy z uczestników projektu otrzymał zadania do wykonania:
\begin{itemize}
\item Domagała Bartosz - zapis sprawozdania ze spotkania
\item Kornata Jarosław - zapis informacji o rozwiązaniu Docker oraz opis przeprowadzonych testów jego konfiguracji
\item Modzelewski Jędrzej - zapis informacji o platformie, języku i bibliotekach używanych w projekcie
\item Stepnowski Marcin - zapis informacji o projekcie architektury
\end{itemize}
\end{itemize}


\subsection[Spotkanie III - 15 kwietnia 2015]{Spotkanie III - 15 kwietnia 2015}

\begin{itemize}
\item Domagała Bartosz - \textit{obecny}
\item Kornata Jarosław - \textit{obecny}
\item Modzelewski Jędrzej - \textit{obecny}
\item Stepnowski Marcin - \textit{obecny}
\end{itemize}

\par{Spotkanie podsumowujące ostatnio wykonane zadania oraz występujące problemy. Zastanowiono się podczas niego nad dotychczasowym wyborem rozwiązań, sprecyzowano treść zadania, ustalono wymagania funkcjonalne i niefunkcjonalne projektu.}


\subsubsection*[Szczegółowy opis wykonanej pracy]{Szczegółowy opis wykonanej pracy}

\begin{itemize}



\item Każdy z uczestników projektu otrzymał zadania do wykonania:
\begin{itemize}
\item Domagała Bartosz - zapis sprawozdania ze spotkania, poprawienie błędów w dokumentacji
\item Kornata Jarosław - 
\item Modzelewski Jędrzej - 
\item Stepnowski Marcin - 
\end{itemize}
\end{itemize}
\chapter{Narzędzia i zewnętrzne biblioteki}

\section{Docker}

\section{Biblioteka ZeroMQ}

\section{Biblioteka protobuf}
\chapter{Ważne zmiany w dokumentacji}

\section{Zmiany w stosunku do dokumentacji etapu I}

\subsubsection*{5.04.2015r.}
\begin{itemize}
\item Poprawa błędu związanego z wymaganiem WF4 oraz z założeniem ZO4 - zmiana dotyczy pomyłki związanej z \textbf{niepełną redundancją} - miała ona dotyczyć warstwy \textbf{wewnętrznej}.
\item Zmiana w założeniu ZO8 - ze względu na funkcjonalność systemu i pozostałe założenia lepiej będzie zrobić asynchroniczną komunikację klienta z serwerem
\end{itemize}


\appendix

% tutaj załączniki
%\chapter[Bibliografia][Bibliografia]{Bibliografia}
\nocite{*}
%\bibliographystylebk{plplain}
%\bibliographystylest{plplain}
%\bibliographystyledoc{plplain}
% \bibliographystyleweb{plplain}
%\bibliographybk{BIB/books}
%\bibliographyst{BIB/books}
%\bibliographydoc{BIB/books}
% \bibliographyweb{BIB/books}


\end{document}




% ex: set tabstop=4 shiftwidth=4 softtabstop=4 noexpandtab fileformat=unix filetype=tex spelllang=pl,en spell:

