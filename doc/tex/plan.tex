\chapter[Organizacja projektu]{Organizacja projektu}
\section[Podział zadań]{Podział zadań}

\par{\textbf{Kierownik projektu, Dokumentalista} - Domagała Bartosz}

\par{\textit{Odpowiedzialny za planowanie, realizację oraz zamykanie projektu. Ma zapewnić osiągnięcie założonych celów projektu i wytworzenie oprogramowania spełniającego określone wymagania jakościowe. Ponadto odpowiedzialny za całą dokumentację projektu, nadzorujący jej tworzenie i ostateczną formę.}}



\par{\textbf{Handlowiec, Specjalista ds. Docker'a, Programista} - Kornata Jarosław}
\par{\textit{Odpowiedzialny za prezentację projektu przed prowadzącym. Ponadto główny specjalista rozwiązania Docker, posiadający na ten temat największą wiedzę, przekazywaną w trakcie projektu pozostałym osobom. Bierze czynny udział w tworzeniu kodu źródłowego programu oraz sumiennie dokumentuje swoją pracę.}}


\par{\textbf{Tester, Programista} - Jędrzej Modzelewski}
\par{\textit{Odpowiedzialny za wszelkie testy związane z oprogramowaniem - projektujący je, programujący oraz egzekwujący. Bierze czynny udział w tworzeniu kodu źródłowego programu oraz sumiennie dokumentuje swoją pracę.}}

\par{\textbf{Architekt, Osoba odp. za repozytorium, Programista} - Stepnowski Marcin}
\par{\textit{Odpowiedzialny za stworzenie i dbanie o odpowiednie wdrożenie architektury projektu oraz dba o zgodność tworzonych rozwiązań informatycznych z obowiązującymi standardami, wzorcami i strategią. Ponadto odpowiedzialny za założenie, zarządzanie oraz tworzenie kopii zapasowych repozytorium projektu. Bierze czynny udział w tworzeniu kodu źródłowego programu oraz sumiennie dokumentuje swoją pracę.}}


\section[Planowany harmonogram pracy]{Planowany harmonogram pracy}

\par{\textbf{Spotkanie I - koniec marca}}

\begin{itemize}
\item Spotkanie organizacyjne
\item Ustalenie ról w projekcie i podział zadań
\item Rozmowa na temat wizji projektu
\item Umówienie się co do organizacji pracy i spotkań projektowych
\end{itemize}

\par{\textbf{Spotkanie II - 1 tydzień kwietnia}}

\begin{itemize}
\item Nauka i zrozumienie zasad działania Docker'a
\item Wspólne sprawdzenie testowych konfiguracji Dockera
\item Ustalenie wstępnej architektury systemu
\item Ustalenie wymagań projektu
\item Ustalenie języka w jakim projekt ma być tworzony
\item Rozmowa na temat dodatkowych narzędzi do realizacji projektu
\end{itemize}


\par{\textbf{Spotkanie III - 2 tydzień kwietnia}}

\begin{itemize}
\item Sprawdzenie i podsumowanie dotychczasowej pracy
\item Rozważanie na temat potencjalnych wad systemu i możliwych sytuacji awaryjnych
\item Ostateczna definicja wymagań funkcjonalnych i niefunkcjonalnych programu
\end{itemize}


\par{\textbf{Spotkanie IV - 3 tydzień kwietnia}}

\begin{itemize}
\item Sprawdzenie i podsumowanie dotychczasowej pracy
\item Upewnienie się o poprawności dotychczasowych założeń, wymagań
\item Wprowadzenie poprawek do dokumentacji
\item Opracowanie ostatecznej dokumentacji na prezentację etapu 1
\end{itemize}

\par{\textbf{Prezentacja etau I - 23 kwietnia}}

\par{\textbf{Spotkanie VI - 4 tydzień kwietnia}}

\begin{itemize}
\item Wprowadzenie poprawek i sugestii Prowadzącego po prezentacji etapu I
\item Podział pracy przy pisaniu kodu źródłowego programu
\item Ustalenie i doprecyzowanie funkcjonalności na prezentację etapu II
\item Rozpoczęcie pisania kodu źródłowego
\item Dyskusja na temat testów systemu i ich definicja
\item Szczegółowe doprecyzowanie rozwiązania
\end{itemize}

\par{\textbf{Spotkanie VII - 4 tydzień kwietnia}}

\begin{itemize}
\item Wykonanie czynności i kwestii, które nie zostały zakończone bądź poruszone podczas spotkania VI
\item Praca nad kodem źródłowym programu i poprawa błędów
\item Testy oprogramowania, przygotowanie do prezentacji funkcjonalności
\end{itemize}

\par{\textbf{Spotkanie VIII - 1 tydzień maja}}

\begin{itemize}
\item Sprawdzenie i podsumowanie dotychczasowej pracy
\item Zdefiniowanie pełnego planu testów i upewnienie się o jego poprawności
\item Ostateczne poprawki do dokumentacji dla etapu II
\item Ostateczne przygotowanie prezentacji funkcjonalności systemu dla etapu II
\end{itemize}

\par{\textbf{Prezentacja etau II - 7 maja}}

\par{\textbf{Spotkanie IX - 2 tydzień maja}}

\begin{itemize}
\item Wprowadzenie poprawek i sugestii Prowadzącego po prezentacji etapu II
\item Praca nad kodem źródłowym programu i poprawa błędów
\item Ustalenie pozostałych zadań do zrobienia
\end{itemize}

\par{\textbf{Spotkanie X - 3 tydzień maja}}

\begin{itemize}
\item Sprawdzenie i podsumowanie dotychczasowej pracy
\item Dyskusja na temat dalszego planu tworzenia kodu źródłowego systemu
\item Praca nad kodem źródłowym programu i poprawa błędów
\end{itemize}

\par{\textbf{Spotkanie XI - 4 tydzień maja}}

\begin{itemize}
\item Sprawdzenie i podsumowanie dotychczasowej pracy
\item Praca nad kodem źródłowym programu i poprawa błędów
\item Opracowanie wstępnej prezentacji końcowej
\end{itemize}

\par{\textbf{Spotkanie XII - 1 tydzień czerwca}}

\begin{itemize}
\item Sprawdzenie i podsumowanie dotychczasowej pracy
\item Praca nad kodem źródłowym programu i poprawa błędów
\item Poprawki do prezentacji końcowej
\end{itemize}

\par{\textbf{Spotkanie XIII - 1 tydzień czerwca}}

\begin{itemize}
\item Spotkanie awaryjne, pozwalajace na wykonanie zaległych czy niedokończonych zadań z poprzednich spotkań
\end{itemize}

\par{\textbf{Spotkanie XIV - 2 tydzień czerwca}}

\begin{itemize}
\item Sprawdzenie i podsumowanie dotychczasowej pracy
\item Ostateczne poprawki do dokumentacji dla etapu III
\item Ostateczne przygotowanie prezentacji funkcjonalności systemu dla etapu III
\end{itemize}

\par{\textbf{Spotkanie XV - 2 tydzień czerwca}}

\begin{itemize}
\item Spotkanie awaryjne, pozwalajace na wykonanie zaległych czy niedokończonych zadań z poprzednich spotkań
\end{itemize}

\par{\textbf{Prezentacja etapu III - 11 czerwca}}

\par{\textbf{Spotkanie XVI - 11 czerwca}}
\begin{itemize}
\item Świętowanie zaliczenia projektu przy piwie
\end{itemize}

\section[Sprawozdania ze spotkań]{Sprawozdania ze spotkań}

\par{Sprawozdania z konkretnych spotkań zawierają listę ustaleń oraz wykonanych czynności w związku z projektem. Poza spotkaniami, każdy z uczestników projektu pracował nad przydzielonymi zadaniami indywidualnie, we własnym zakresie, trzymając się konkretnych terminów.}

\subsection[Spotkanie I - 31 marca 2015]{Spotkanie I - 31 marca 2015}

\begin{itemize}
\item Domagała Bartosz - \textit{obecny}
\item Kornata Jarosław - \textit{obecny}
\item Modzelewski Jędrzej - \textit{obecny}
\item Stepnowski Marcin - \textit{obecny}
\end{itemize}

\par{Spotkanie organizacyjne, którego celem był podział ról w projekcie oraz przydzielenie zadań, a także wstępne omówienie samej wizji projektu.}


\subsubsection*[Szczegółowy opis wykonanej pracy]{Szczegółowy opis wykonanej pracy}

\begin{itemize}
\item Kierownik projektu ustalił podział ról zgodnie z umiejętnościami i preferencjami osób z grupy projektowej:

\begin{itemize}
\item Domagała Bartosz - Kierownik, Dokumentalista
\item Kornata Jarosław - Handlowiec, Specjalista ds. Docker'a, Programista
\item Modzelewski Jędrzej - Tester, Programista
\item Stepnowski Marcin - Architekt, Osoba odpowiedzialna za repozytorium, Programista
\end{itemize}

\item Ustalony został termin cotygodniowych spotkań projektowych
\item Omówiona została treść zadania, upewniono się, że każdy z uczestników projektu je rozumie
\item Prowadzone były rozmowy na temat wizji projektu każdego z uczestników, dobre pomysły spisywane i komentowane
\item Zastanowiono się nad sprecyzowaniem treści zadania
\item Każdy z uczestników projektu otrzymał zadania do wykonania:
\begin{itemize}
\item Domagała Bartosz - stworzenie szkieletu dokumentacji za pomocą LateXa, zapisanie sprawozdania z odbytego spotkania
\item Kornata Jarosław - dowiedzenie się jak najwięcej o rozwiązaniu Docker
\item Modzelewski Jędrzej - pomyślenie o narzędziach, bibliotekach i językach programowania jakich można użyć w projekcie
\item Stepnowski Marcin - stworzenie repozytorium Git oraz zastanowienie się nad architekturą projektu
\end{itemize}
\end{itemize}

\subsection[Spotkanie II - 8 kwietnia 2015]{Spotkanie II - 8 kwietnia 2015}

\begin{itemize}
\item Domagała Bartosz - \textit{obecny}
\item Kornata Jarosław - \textit{obecny}
\item Modzelewski Jędrzej - \textit{obecny}
\item Stepnowski Marcin - \textit{obecny}
\end{itemize}

\par{Spotkanie podsumowujące ostatnio wykonane zadania oraz występujące problemy. Ponadto podczas niego ustalono wstępną treść zadania i omówiono, a także przetestowano rozwiązanie Docker.}


\subsubsection*[Szczegółowy opis wykonanej pracy]{Szczegółowy opis wykonanej pracy}

\begin{itemize}
\item Przekazanie przez specjalistę ds. Dockera informacji o tym rozwiązaniu pozostałym uczestnikom
\item Wspólne sprawdzenie testowych konfiguracji Dockera
\item Wstępny projekt architektury projektu zaproponowany przez Architekta
\item Zastanowienie się nad wymaganiami niefunkcjonalnymi projektu
\item Ustalenie języka w jakim stworzony zostanie projekt - Java
\item Rozmowa na temat przydatnych i potrzebnych bibliotek w projekcie

\item Każdy z uczestników projektu otrzymał zadania do wykonania:
\begin{itemize}
\item Domagała Bartosz - zapis sprawozdania ze spotkania
\item Kornata Jarosław - zapis informacji o rozwiązaniu Docker oraz opis przeprowadzonych testów jego konfiguracji
\item Modzelewski Jędrzej - zapis informacji o platformie, języku i bibliotekach używanych w projekcie
\item Stepnowski Marcin - zapis informacji o projekcie architektury
\end{itemize}
\end{itemize}

\subsection[Spotkanie X1 - 14 kwietnia 2015]{Spotkanie X1 - 14 kwietnia 2015}

\begin{itemize}
\item Domagała Bartosz - \textit{obecny}
\item Kornata Jarosław - \textit{nieobecny}
\item Modzelewski Jędrzej - \textit{nieobecny}
\item Stepnowski Marcin - \textit{obecny}
\end{itemize}

\par{Pierwsze spotkanie z Prowadzącym projekt. Informacje ze spotkania:}

\begin{itemize}
\item Zatwierdzenie przez Prowadzącego tematu
\item Zgoda Prowadzącego na brak funkcji logowania w aplikacji klienckiej, ze względu na główny cel projektu, ograniczony czas i mniejszą liczbę osób w grupie niż przewidziana
\item Zgoda Prowadzącego na brak graficznego interfejsu użytkownika, ze względu na główny cel projektu, ograniczony czas i mniejszą liczbę osób w grupie niż przewidziana
\item Zgoda Prowadzącego na brak interaktywnej aplikacji klienckiej, ze względu na główny cel projektu, ograniczony czas i mniejszą liczbę osób w grupie niż przewidziana
\item Doprecyzowanie szczegółów w związku z 3-warstwową naturą systemu
\item Uzyskanie odpowiedzi na pytania odnośnie architektury
\end{itemize}

\subsection[Spotkanie III - 14 kwietnia 2015]{Spotkanie III - 14 kwietnia 2015}

\begin{itemize}
\item Domagała Bartosz - \textit{obecny}
\item Kornata Jarosław - \textit{obecny}
\item Modzelewski Jędrzej - \textit{obecny}
\item Stepnowski Marcin - \textit{obecny}
\end{itemize}

\par{Spotkanie podsumowujące ostatnio wykonane zadania oraz występujące problemy. Zastanowiono się podczas niego nad dotychczasowym wyborem rozwiązań, sprecyzowano treść zadania, ustalono wymagania funkcjonalne i niefunkcjonalne projektu.}

\subsubsection*[Szczegółowy opis wykonanej pracy]{Szczegółowy opis wykonanej pracy}

\begin{itemize}
\item Dyskusja na temat założeń systemu i jego wstępnej architektury
\item Wprowadzenie poprawek do projektu na podstawie wniosków z dyskusji
\item Wspólne ustalenie dodatkowych założeń, przesłanek do ich egzekwowania i konsekwencji z tym związanych
\item Definicja wymagań funkcjonalnych i niefunkcjonalnych programu
\item Rozmowa na temat wad systemu, potencjalnych zagrożeń, spisanie pomysłów

\item Każdy z uczestników projektu otrzymał zadania do wykonania:
\begin{itemize}
\item Domagała Bartosz - zapis sprawozdania ze spotkania, poprawienie błędów w dokumentacji oraz edycja szkieletu stylu, upewnienie się, że naniesione przez resztę grupy poprawki są prawidłowe
\item Kornata Jarosław - wprowadzenie poprawek do opisu przykładowej konfiguracji Dockera
\item Modzelewski Jędrzej - wprowadzenie poprawek do opisu języka oraz IDE w jakim tworzony będzie projekt
\item Stepnowski Marcin - rozwinięcie i wyjaśnienie decyzji projektowych dotyczących architektury podjętych w trakcie spotkania oraz zapisanie ich w dokumentacji
\end{itemize}
\end{itemize}


\subsection[Spotkanie X2 - 21 kwietnia 2015]{Spotkanie X2 - 21 kwietnia 2015}

\begin{itemize}
\item Domagała Bartosz - \textit{obecny}
\item Kornata Jarosław - \textit{nieobecny}
\item Modzelewski Jędrzej - \textit{nieobecny}
\item Stepnowski Marcin - \textit{obecny}
\end{itemize}

\par{Drugie spotkanie z Prowadzącym projekt. Informacje ze spotkania:}

\begin{itemize}
\item Zgoda Prowadzącego na pełną redundancję w warstwie wewnętrznej serwera (tzn. każdy z węzłów ma posiadać wszystkei aktualne dane) w związku z ograniczonym czasem i mniejszą liczbą osób w grupie projektowej niż przewidziana
\item Zdobycie informacji odnośnie warstw serwera, które mogą być na tym samym węźle jak i na dwóch niezależnych węzłach
\item Zgoda Prowadzącego na użycie protokołu TCP/IP do komunkacji między warstwami serwera bez względu na ich lokalizację - w związku z ograniczonym czasem i mniejszym skomplikowaniem zagadnienia
\end{itemize}

\subsection[Spotkanie IV - 22 kwietnia 2015]{Spotkanie IV (TeamSpeak) - 22 kwietnia 2015}

\begin{itemize}
\item Domagała Bartosz - \textit{obecny}
\item Kornata Jarosław - \textit{obecny}
\item Modzelewski Jędrzej - \textit{nieobecny}
\item Stepnowski Marcin - \textit{obecny}
\end{itemize}

\par{Spotkanie za pośrednictwem programu TeamSpeak mające na celu podsumowanie dotychczasowej pracy oraz wprowadzenie ostatecznych poprawek do dokumentacji przed prezentacją etapu I projektu.}

\subsubsection*[Szczegółowy opis wykonanej pracy]{Szczegółowy opis wykonanej pracy}
\begin{itemize}
\item Poprawienie błędów w dokumentacji
\item Dopisanie brakujących rzeczy do dokumentacji (głównie - kwestii potencjalnych problemów z systemem)
\item Ostateczna edycja i poprawki szablonu projektu
\item Podsumowanie dotychczasowej pracy, umówienie się na kolejne spotkanie w celu zaczęcia pracy nad kodem źródłowym systemu
\item Rozmowa i ustalenia na temat prezentacji etapu I
\end{itemize}

\subsection[Spotkanie V - 27 kwietnia 2015]{Spotkanie V - 27 kwietnia 2015}

\begin{itemize}
\item Domagała Bartosz - \textit{obecny}
\item Kornata Jarosław - \textit{obecny}
\item Modzelewski Jędrzej - \textit{obecny}
\item Stepnowski Marcin - \textit{obecny (TeamSpeak)}
\end{itemize}

\par{Pierwsze spotkanie po oddaniu etapu I projektu. Omówienie niezbędnych poprawek do dotychczasowej dokumentacji oraz opracowanie planu dalszej pracy związanej z kodem źródłowym programu i prezentacją.} 

\subsubsection*[Szczegółowy opis wykonanej pracy]{Szczegółowy opis wykonanej pracy}
\begin{itemize}
\item Poprawki w dokumentacji związane z sugestiami od Prowadzącego
\item Opracowanie wstępnego planu testów
\item Podział obowiązków przy pisaniu kodu źródłowego programu:
\begin{itemize}
\item Kornata Jarosław - struktura wiadomości przesyłanych między warstwami (oraz klasy pomocnicze); obiekty i funkcje liczące dane statystyczne w warstwie zewnętrznej; klasy do komunikacji z bazą danych, program do generowania danych wrażliwych
\item Modzelewski Jędrzej - komunikacja wewnętrzna serwerów wewnętrznych, komunikacja między warstwami
\item Stepnowski Marcin - komunikacja wewnętrzna serwerów zewnętrznych, komunikacja między warstwami
\end{itemize}
\item Opracowanie wstępnego scenariusza prezentacji etapu II projektu
\item Ustalenie zastosowania wzorca czynnościowego wzorca projektowego przy komunikacji między warstwami zwanego łańcuchem zobowiązań
\end{itemize}