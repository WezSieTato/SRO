
\begin{titlepage}
    % Strona tytułowa
    \vbox to\textheight{\hyphenpenalty=10000
    \begin{center}
	\begin{tabular}{p{107mm} p{9cm}}
	    \begin{minipage}{9cm}
	      \begin{center}
	      Politechnika Warszawska \\
	      Wydział Elektroniki i~Technik Informacyjnych \\
	      Instytut Informatyki
	      \end{center}
	    \end{minipage}
	    &
	    \begin{minipage}{8cm}
	    \begin{flushleft}
	     \footnotesize
	      Rok akademicki 2013/2014
	    \vspace*{2.75\baselineskip}
	    \end{flushleft}
	    \end{minipage} \\
	\end{tabular}
	\vspace*{3.75\baselineskip}

		
	\par\vspace{\smallskipamount}
	\vspace*{2\baselineskip}{\LARGE PRACA DYPLOMOWA INŻYNIERSKA\par}
	\vspace{3\baselineskip}{\LARGE\strut Marcin Stepnowski\par}
	\vspace*{2\baselineskip}{\huge\bfseries Rozbudowa aplikacji SpriteBuilder o wsparcie dla silnika Cocos2d-x\par}

	\vspace*{7\baselineskip}
	\hfill\mbox{}\par\vspace*{\baselineskip}\noindent
	\begin{tabular}[b]{@{}p{3cm}@{\ }l@{}}
	    {\large\hfill } & {\large }
	\end{tabular}
	\hfill
	\begin{tabular}[b]{@{}l@{}}
	Opiekun pracy: \\[\smallskipamount]
	{\large mgr inż. Waldemar Grabski}
	\end{tabular}\par
	\vspace*{4\baselineskip}
    \begin{tabular}{p{\textwidth}}
    \begin{flushleft}
	\begin{minipage}{7cm}
	Ocena \dotfill
	\par\vspace{1.6\baselineskip}
	\dotfill
	\par\noindent
	\centerline{\footnotesize Podpis Przewodniczącego} \par
	\centerline{\footnotesize Komisji Egzaminu Dyplomowego}\par
	\end{minipage}
    \end{flushleft}
    \end{tabular}
    \end{center}}

    % Życiorys
    \newpage\thispagestyle{empty}
    \begin{tabular}{p{5cm} p{12cm}}
    \begin{minipage}{5cm}
    \center
    \end{minipage}
    &
    \begin{minipage}{12cm}
    \begin{flushleft}
    \par\noindent\vspace{1\baselineskip}
    \renewcommand{\arraystretch}{1.5}\begin{tabular}[h]{l l}
    {\normalsize\it Kierunek:} & Informatyka \\
    {\normalsize\it Specjalność:} & Inżynieria Systemów Informatycznych \\
    \end{tabular}
    \par\noindent\vspace{1\baselineskip}
    \renewcommand{\arraystretch}{1.5}\begin{tabular}[h]{l l}
    {\normalsize\it Data urodzenia:} & {\normalsize 31 stycznia 1991~r.} \\
    {\normalsize\it Data rozpoczęcia studiów:} & {\normalsize 21 lutego 2011 r.}
    \end{tabular}
    \par\noindent\vspace{1\baselineskip}
    \end{flushleft}
    \end{minipage}
    \end{tabular}
    \vspace*{1\baselineskip}
    \begin{center}
	{\large\bfseries Życiorys}\par\bigskip
    \end{center}

    \indent
    
\begin{small}
    Urodziłem się 31 stycznia 1991 r. w Ostrołęce. W latach 2008-2010 uczęszczałem do II LO im. Kamila Cypriana Norwida w Ostrołęce, do klasy o profilu matematyczno-informatyczno-językowym. W lutym 2011 r. rozpoczączełem studia dzienne pierwszego stopnia na kierunku Informatyka na Wydziale Elektroniki i Technik Informacyjnych Politechniki Warszawskiej. 
    
    W lipcu 2012 podjąłem pracę jako programista aplikacji mobilnych w Fundacji Festina Lente. Współpracowałem także z Instytutem Badań Stosowanych Politechniki Warszawskiej tworząc aplikację mobilną na systemy iOS. Dziedzinami wiedzy, które najbardziej mnie pasjonują na moim kierunku są gry komputerowe, aplikacje mobilne, wzorce projektowe oraz stosowanie przejrzystego stylu implementacji.
    Poza studiami i pracą interesuję się muzyką rockową z lat 70 oraz prowadzę małą pasiekę pszczelarską umiejscowioną w Aleksandrowie, niedaleko Ostrołęki.
\end{small}
    
    \par
    \vspace{2\baselineskip}
    \hfill\parbox{15em}{{\small\dotfill}\\[-.3ex]
    \centerline{\footnotesize podpis studenta}}\par
    \vspace{3\baselineskip}
    \begin{center}
 	{\large\bfseries Egzamin dyplomowy} \par\bigskip\bigskip
    \end{center}
    \par\noindent\vspace{1.5\baselineskip}
    Złożył egzamin dyplomowy w dn. \dotfill
    \par\noindent\vspace{1.5\baselineskip}
    Z wynikiem \dotfill
    \par\noindent\vspace{1.5\baselineskip}
    Ogólny wynik studiów \dotfill
    \par\noindent\vspace{1.5\baselineskip}
    Dodatkowe wnioski i uwagi Komisji \dotfill
    \par\noindent\vspace{1.5\baselineskip}
    \dotfill

    % Streszczenie
    \newpage\thispagestyle{empty}
    \vspace*{2\baselineskip}
    \begin{center}
	{\large\bfseries Streszczenie}\par\bigskip
    \end{center}

    {\itshape
    Niniejsza praca poświęcona jest opracowaniu rozszerzenia dla programu SpriteBuilder, które pozwoliłoby na importowanie stworzonego graficznego interfejsu użytkownika do kodu źródłowego w języku C++ z wykorzystaniem silnika do tworzenia gier Cocos2d-x. Program ten jest dostępny na platformę Mac OS X, jednak generowany możemy skompilować i uruchomić na większości obecnie dostępnych systemach operacyjnych (iOS, Android, Windows, Windows Phone, Linux, Mac OS X, a także inne). Praca została zaprojektowana z myślą o łatwym i prostym wprowadzaniu zmian do projektu w przyszłości. }
    \vspace*{1\baselineskip}

    \noindent{\bf Słowa kluczowe}: {\itshape graficzny interfejs użytkownika, silnik gry, edytor}
    \par
    \vspace{4\baselineskip}
    \begin{center}
	{\large\bfseries Abstract}\par\bigskip
    \end{center}
    \noindent{\bf Title}: {\itshape Adding support for cocos2d-x game engine to SpriteBuilder app}\par
    \vspace*{1\baselineskip}
    {\itshape
    The aim of this thesis is elaborate the extension of SpireBuilder aplication, which allow you to import just created graphical user interface to the source code in C++ with using game engine Cocos2d-x. This program is accessible on Mac OS X platform, however generated code we can compile and run on currently available operation systems (iOS, Android, Windows, Windows Phone, Linux, Mac OS C and more).
    }
    \vspace*{1\baselineskip}

    \noindent{\bf Key words}: {\itshape GUI, game engine, editor}

\end{titlepage}

% ex: set tabstop=4 shiftwidth=4 softtabstop=4 noexpandtab fileformat=unix filetype=tex spelllang=pl,en spell:
